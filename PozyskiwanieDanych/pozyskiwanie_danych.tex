\documentclass{article}
\usepackage[polish]{babel}
\usepackage[utf8]{inputenc}
\usepackage{polski}
\usepackage{hyperref}
\frenchspacing
\setcounter{tocdepth}{2}
\usepackage{graphicx}
\graphicspath{ {images/} }
\usepackage{float}

\begin{document}
	
\begin{titlepage}

\newcommand{\HRule}{\rule{\linewidth}{0.5mm}}

\center

%----------------------------------------------------------------------------------------

\textsc{\LARGE Politechnika Warszawska}\\[10mm]
\textsc{\LARGE Wydział Matematyki i Nauk}\\[4mm]
\textsc{\LARGE Informacyjnych}\\[4cm]
 
%----------------------------------------------------------------------------------------

\textsc{\Huge Metody Data Science}\\[0.5cm]

%----------------------------------------------------------------------------------------

\HRule \\[0.4cm]
{ \LARGE \bfseries Moduł pozyskiwania danych}\\[5.0cm]
 
 
%----------------------------------------------------------------------------------------

\begin{flushright}
\Large \emph{Autorzy:}\\[0.5cm]
Piotr \textsc{Izert}\\
Przemysław \textsc{Rząd}\\
Anna \textsc{Zawadzka}\\
\end{flushright}

%----------------------------------------------------------------------------------------

\vfill
{\large \today}\\[3cm]

\end{titlepage}
	
\newpage

\section{Wstęp}

Niniejszy projekt ma na celu wykonanie uczącego się systemu, który na podstawie pobieranych danych po odpowiednim ich przetworzeniu przygotowuje analizę w postaci prognozowania oraz  klasyfikacji. Pierwszym etapem projektu jest stworzenie modułu pozyskiwania danych.

%----------------------------------------------------------------------------------------

\section{Dane}

Źródłem danych są zbiory udostepnione przez ministerstwo transportu rządu Wielkiej Brytanii na stronach internetowych:
\begin{itemize}
    \item \url{https://data.gov.uk/dataset/road-accidents-safety-data}
    \item \url{https://data.gov.uk/dataset/dft-eng-srn-routes-journey-times}
\end{itemize}
Dane obejmują informacje na temat wypadków drogowych na terenie Wielkiej Brytanii oraz średnich prędkościach przejazdu samochodów na danych odcinkach dróg w latach 2009-2014. Przechowywane są w plikach o formacie CSV. Dane pobierane są jako zamknięty zbiór rekordów, natomiast wykorzystane będą jako dane napływające w czasie rzeczywistym. 


%----------------------------------------------------------------------------------------

\section{Moduł pozyskiwania danych}

Użyte narzędzia:
\begin{itemize}
    \item maszyna wirtualna Hortonworks Sandbox (HDP 2.4)
    \item Flume
    \item Spark
    \item HDFS
\end{itemize}

\begin{figure}[H]
\centering
\includegraphics[scale=0.6]{schemat}
\caption{Schemat modułu pozyskiwania danych}
\end{figure}

W katalogu \textit{rawData} na maszynie wirtualnej umieszczone zostaną pliki w formacie CSV pięciu kategorii: wypadki, samochody, marki i modele samochodów, ofiary oraz średnie prędkości na poszczególnych odcinkach dróg. Dane o wypadkach, ofiarach, samochodach oraz ich markach i modelach pogrupowane są w pliki ze względu na rok, natomiast dla danych o prędkościach istnieją osobne pliki dla każdego miesiąca w danym roku.

Następnie za pomocą skryptu napisanego w języku Python zostaną połączone pliki z danymi o wypadkach, ofiarach, samochodach oraz ich markach i modelach dla każdego roku (za pomocą instrukcji SQL). Pliki wynikowe (osobne dla każdego roku) będą zawierać dane o wypadkach, samochodach w nich uczestniczących wraz z informacją o markach i modelach oraz ofiarach tychże wypadków.

Do każdego wypadku przyporządkowanych jest zawsze kilka samochodów, natomiast nie dla wszystkich pojazdów istnieje informacja o ofiarach (poszkodowani przyporządkowani są do konkretnego pojazdu) oraz o marce i modelu.

Nowo utworzone pliki zostaną automatycznie skopiowane do katalogu \textit{spoolDir1} (\textit{spooling directory}), który jest specjalnym katalogiem monitorowanym przez Flume’a. Dane o prędkościach przejazdów będą automatycznie umieszczane w katalogu \textit{spoolDir2} bez żadnych zmian. Jeżeli w katalogach \textit{spooling} pojawiają się w nim nowe pliki, Flume rozpoczyna ich przetwarzanie.

Konfiguracja Flume’a zakłada istnienie dwóch źródeł (\textit{source}), którymi są \textit{spoolDir1} i \textit{spoolDir2}, oraz czterech ujść (\textit{sink}), po dwa dla HDFS’a i dla Spark’a, odpowiednio dla skonsolidowanych danych o wypadkach oraz dla danych o prędkościach przejazdów. Ujścia skierowane do HDFS’a odwoływać się będą do katalogów \textit{Accidents} oraz \textit{Speeds}, natomiast ujścia do Spark’a odwołują się do odpowiedniego Job’a, który przetworzy dane w sposób strumieniowy.

W systemie HDFS istnieje dodatkowo katalog \textit{Dictionaries}, w którym umieszczone są pliki słownikowe, wykorzystywane w dalszych etapach projektu. Pliki w tym katalogu umieszczone będą ręcznie i jednorazowo.  

Dane ze wszystkich katalogów w HDFS (\textit{Accidents}, \textit{Speeds}, \textit{Dictionaries}) pobierane będą przez Job’a (Spark) w sposób wsadowy i używane do generowania prognoz.

Przetwarzanie strumieniowe także korzystać będzie z danych znajdujących się w HDFS w katalogu Dictionaries i będzie realizować zadanie klasyfikacji.





\end{document}



